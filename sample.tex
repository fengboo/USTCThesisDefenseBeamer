\documentclass[10pt]{beamer}	%打印幻灯片
%\documentclass[10pt,handout]{beamer}	%打印手稿
\usepackage{style}
\addbibresource{sample_ref.bib}
\usepackage{pgfpages}	%打印6合1用
%\pgfpagesuselayout{6 on 1}[a4paper,border shrink=5mm]	%打印6合1用
%\setbeameroption{show notes on second screen=right}	%在PDF右侧打印注释\note{}
\setbeameroption{hide notes}

%%%%%%%%%%%%%%%
%%打印幻灯片带有动画
%%打印手稿没有动画
%%打印6合1会在同一页显示6页幻灯片,也可以将6 on 1改为2 on 1等
%%%%%%%%%%%%%%%

%定义两个蓝色命令,用于强调显示
\newcommand{\Blue}{\bfseries \color{njuPurple}}
\newcommand{\blue}{\color{njuPurple}}

\begin{document}
\setbeamertemplate{footline}{}
\title[炁体源流与天师度]{炁体源流与天师度\\的关键技术研究}	%方括号内为页脚中的简写标题,花括号内为标题,标题控制在一至两行,两行中用双右斜线\\分开
\author{张楚岚}
\supervisor{张之维~教授}
\id{BA20200020}
\institute{一人之下}
\date{\today}

\frame{\titlepage}

\begin{frame}
\frametitle{目录}
\makecontents
\end{frame}

\section{研究背景}
\begin{frame}[fragile]
\frametitle{研究背景}
国外很牛B,中国很牛B,我研究的内容很牛B。\cite{wootters1982single}

虽然可以加参考文献,不过一般{\Blue 不建议}在幻灯片中加参考文献,会分散听众的注意力。

这里演示一下{\Blue Blue}和{\blue blue}的效果。

使用方法分别为\verb|{\Blue <文字>}|和\verb|{\blue <文字>}|。
\end{frame}

\begin{frame}
\frametitle{异人发展趋势}
这里介绍一下columns和block的用法。
\begin{columns}[t]
\begin{column}{0.33\textwidth}
\begin{block}{标准化}
目前国内外异人尚无统一的标准。猫猫狗狗都称自己异人。建立标准的异人体系非常重要。
\end{block}
\end{column}
\begin{column}{0.33\textwidth}
\begin{block}{现代化}
虽然天师府等已经融入现代社会,但有些山沟里仍有些与现代社会脱节的练炁世家。
\end{block}
\end{column}
\begin{column}{0.33\textwidth}
\begin{block}{健康化}
涉及鬼尸妖魔的异人不符合社会发展的主流价值观,应该加强教育引导。
\end{block}
\end{column}
\end{columns}
\end{frame}

\begin{frame}[fragile]
\frametitle{关于目录}
\begin{itemize}
\item 目录为{\blue 二级目录},一级目录section控制在四至六个,二级目录subsection控制在四个以内。
\item 每一页幻灯片的标题放在\verb|\frametitle{<标题>}|内,不作为目录。
也可以直接放在\verb|\begin{frame}{<标题>}|内。
\item 如果在幻灯片中使用了\verb|\verb|代码,要在\verb|\begin{frame}|后增加\verb|[fragile]|,即\verb|\begin{frame}[fragile]|
\end{itemize}
\end{frame}

\begin{frame}[fragile]
\frametitle{字体大小}
字体按从小到大分别为

\verb|\tiny \scriptsize \footnotesize \small \normalsize|
\verb|\large \Large \LARGE \huge \Huge|

\tiny 
凡夫俗子张之维

\scriptsize
不摇碧莲张楚岚

\footnotesize
五百一晚贾正亮

\small
一贫如洗王道长

\normalsize
机智一比冯宝宝

\large
不听八卦诸葛青

\Large
一生无暇陆老爷

\LARGE
过大

\huge
用不到

\Huge
真用不到
\end{frame}

\section{第二节}

\subsection{插入图片}

\begin{frame}
\frametitle{图片}
虽然图表可以插入说明,但在图少时不建议加入说明,在图较多时可适当标注。
\begin{figure}
\centering
\includegraphics[width=0.8\textwidth]{ustc_logo_name.png}
\caption{这里是Caption。}
\end{figure}
\end{frame}

\subsection{插入表格}

\begin{frame}
\frametitle{表格}
表格推荐使用三线表。
\begin{table}[htb]
\centering\scriptsize
% \caption{当世异人战斗力对比}
 \label{tab:TSVSOther}
 \begin{tabular}{cccc}
\toprule
姓名 & 所属 & 绝技 & 战斗力(单位:巴掌)\\
\midrule
张之维 & 天师府  & 天师度 & 999\\
丁嵨安 & 全性 & 不明 & 10\\
陆瑾 & 三一门 & 通天箓 & 1\\
王也 & 武当 & 风后奇门 & 0.3\\
\bottomrule
\end{tabular}
\end{table}
\end{frame}

\section{第三节}

\subsection{第三节第一部分}

\begin{frame}
	\frametitle{标题}
\end{frame}

\subsection{第三节第二部分}

\begin{frame}
	\frametitle{标题}
\end{frame}

\subsection{第三节第三部分}

\begin{frame}
	\frametitle{标题}
\end{frame}

\section{第四节}
\begin{frame}
	\frametitle{标题}
\end{frame}

\section{总结和展望}
\begin{frame}
\frametitle{工作总结}
介绍自己的工作
\end{frame}

\begin{frame}
\frametitle{创新点}
\begin{columns}[t]
\begin{column}{0.33\textwidth}
\begin{block}{异人界}
\begin{itemize}
\item 提出异人标准化方案
\item 现代化
\item 指出健康化是异人发展的重要趋势
\end{itemize}
\end{block}
\end{column}
\pause
\begin{column}{0.33\textwidth}
\begin{block}{炁体源流}
\begin{itemize}
\item 提出㤅体源流远程化方案
\item 在通信中的应用
\item 对异人现代化发展的潜力
\end{itemize}
\end{block}
\end{column}
\pause
\begin{column}{0.33\textwidth}
\begin{block}{天师度}
\begin{itemize}
\item 提出天师度在知识传承上的方案
\item 在教学上的应用
\item 对全人类的巨大价值
\end{itemize}
\end{block}
\end{column}
\end{columns}
\end{frame}

\begin{frame}
\frametitle{已发表的论文}
\begin{enumerate}
\item {\Blue{Zhang Chulan}}, Feng Baobao, Xu si, et al. Study on Shougongsha [J]. Review of Yiren, 2020, 30(2): 05532.
\item {\Blue{Zhang Chulan}}, Feng Baobao, Wang Ye, et al. Study on mairen [C]//2020 233th International Conference on Yiren. 2020: 168-179.
\end{enumerate}
\end{frame}

\setbeamertemplate{footline}[footlineoff]%取消页脚
\begin{frame}[noframenumbering]
	\thanks
\end{frame}
\setbeamertemplate{footline}[footlineon]%添加页脚

%\setbeamertemplate{footline}[footlineoff]%取消页脚
\section*{Backup}
\begin{frame}[noframenumbering]{Backup}
幻灯片正篇中用到的数学推导或其他的补充说明。

Backup不在目录上显示,并且不计算在正文的页码中,但可通过底部导航条点击进入。
\end{frame}
%\setbeamertemplate{footline}[footlineon]%添加页脚

\end{document}